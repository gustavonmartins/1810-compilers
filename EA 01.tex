\documentclass[11pt]{scrartcl}
\usepackage[utf8]{inputenc}
\usepackage{mathtools}
\usepackage{amssymb}
\usepackage{caption}
\usepackage{color}
\usepackage{xcolor}
\usepackage{listings}
\DeclareCaptionFont{white}{\color{white}}
\DeclareCaptionFormat{listing}{\colorbox{gray}{\parbox{\textwidth}{#1#2#3}}}
\captionsetup[lstlisting]{format=listing,labelfont=white,textfont=white}
\title{\textbf{1810 Einsendenaufgabe KE 01}}
\author{Gustavo Nunes Martins}
\date{29 September 2018}
\begin{document}
	\maketitle
	\section*{Aufgabe 1}
	\section*{Aufgabe 2}	\begin{lstlisting}[label=some-code,caption=Lexer Code]
	#define TRUE 1
	
	int gettoken(){
	int c;
	state = 0; start_state=0;
	
	while (TRUE){
	switch(state){
	case 11:
	c = nextchar();
	if isdigit(c)		state=12;
	else if issign(c)	state=13;
	else				state=next_diagram();
	break;
	
	case 12:
	c = nextchar();
	if c=='.'			state=14;
	else if isdigit(c)	state=12;
	else				state=next_diagram();
	break;
	
	case 13:
	c=nextchar();
	if isdigit(c)		state=12;
	else				state=next_diagram();
	break;
	
	case 14:
	c=nextchar();
	if isdigit(c)		state=15;
	else if c=='E'		state=17;
	else				state=16;
	break;
	
	case 15:
	c=nextchar();
	if isdigit(c)		state=15;
	else if c=='E'		state=17;
	else				state=16;
	break;
	
	case 16:
	return REAL;
	break;
	
	case 17:
	c=nextchar();
	if isdigit(c)		state=18;
	else if issign(c)	state=19;
	else				state=next_diagram();
	break;
	
	case 18;
	c=nextchar(c)
	if isdigit(c)		state=18;
	else				state=16;
	break;
	
	case 19:
	c=nextchar();
	if isdigit(c)		state=18;
	else				state=next_diagram();
	}
	}
	}
	\end{lstlisting}
	\section*{Aufgabe 3}
	Anmerkungen: 
	\begin{itemize}
		\item Das lesen von negativ-Nummern lauft nicht nur durch den Lexer, aber auch durch den Parser. Das erleichtert die Entscheidung von "-" als negativ-Nummern oder als Subtraktion.
		\item Punkte (z.B. (3,2) oder (p,5)) und Funktionen ((v, f(v)) sind alle durch Tupeln repr\"{a}sentiert.
		\item Relative Bindungskr\"{a}fte zwischen Multiplikation, Division, Summierung und Subtraktion sind durch %token Vorrang verwirklicht.
		\item Der Code fuers Lexer und Parser ist als zip Datei geliefert.
		
	\end{itemize}
\end{document}